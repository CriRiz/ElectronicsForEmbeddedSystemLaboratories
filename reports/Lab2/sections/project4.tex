\subsection{Project \#4}

The project 4's main purpose is to create an UART receiver with software and test it sending characters from computer tool PuTTY on the Cyclone V board.\\

To do so it been first defined the following costans at the beginning of the code:

\begin{listing}[ht]
\begin{minted}[linenos]{c}
#define NBIT        8
#define NSTOPBIT    1
#define NOPARITY    0
#define EVENPARITY  1
#define ODDPARITY   2
#define PARITY      NOPARITY
\end{minted}
\caption{Constants}
\label{lst:Const}
\end{listing}

In this project it was not used a parity bit, so each transmission consists of exactly 10 bits: 1 start bit, 8 data bits, and 1 stop bit.

\subsubsection{Version \#1}

In order to create an UART receiver the focus is on the UART's message structure.

\begin{listing}[ht]
\begin{minted}[linenos]{c}
#define BAUDRATE 300

int main() {

    int ticks_per_sec = alt_timestamp_freq();
    int ticks_per_bit = ticks_per_sec / BAUDRATE;

    int c[NBIT];
    int val;

    printf("UART RX ready.\n");

    while (1) {

        do {
            val = IORD_ALTERA_AVALON_PIO_DATA(NIOS_UARTRX_BASE) & 0x01;
        } while (val == 1);

        alt_timestamp_start();
        while (alt_timestamp() < (ticks_per_bit >> 1)) {}

        val = IORD_ALTERA_AVALON_PIO_DATA(NIOS_UARTRX_BASE) & 0x01;
        if (val != 0) {
            continue;
        }
        alt_timestamp_start();
        int sample_times[NBIT + NSTOPBIT];
        for (int i = 0; i < NBIT + NSTOPBIT; i++)
            sample_times[i] = (i + 1) * ticks_per_bit;

        for (int i = 0; i < NBIT; i++) {
            while (alt_timestamp() < sample_times[i]) {}
            val = IORD_ALTERA_AVALON_PIO_DATA(NIOS_UARTRX_BASE) & 0x01;
            c[i] = val;        // c[0] = LSB
        }

        while (alt_timestamp() < sample_times[NBIT]) {}
        val = IORD_ALTERA_AVALON_PIO_DATA(NIOS_UARTRX_BASE) & 0x01;

        if (val != 1) {
            printf("ERRORE: stop bit non valido!\n");
            continue;

        int result = 0;
        for (int i = 0; i < NBIT; i++)
            // | è l'operatore OR bit a bit
            result |= (c[i] << i);

        printf("Ricevuto: %c (0x%02X) (%d)\n", result, result, result);
    }

    return 0;
}
\end{minted}
\caption{Version\_1}
\label{lst:V1}
\end{listing}

The code first waits for the start bit. The default state of the UART line is 1, so the start bit is represented by 0. 
To ensure that it is not an error, the code rereads the line state after half a bit period (see line 22 of \ref{lst:V1}). 
If it is indeed the start bit, it computes the sampling times and then begins sampling the other 8 data bits of the character encoded in ASCII, as shown in the for loop on line 31.\\
In line 38 in the end it makes sure that the trasmission ends with a stop bit (equal to 1) and then it rebuilds the character (for loop in line 45).\\

\subsection{Version \#2}
In \ref{lst:V1} the code use a for loop to read the 8 char bit trasmitted by UART. The branch instructions used by for loop can slow down execution.\\
For this reason the \ref{lst:V2} uses loop unrolling to improve performance:

\begin{listing}[ht]
\begin{minted}[linenos]{c}
#define BAUDRATE 300

int main() {

    int ticks_per_sec = alt_timestamp_freq();
    int ticks_per_bit = ticks_per_sec / BAUDRATE;

    int c[NBIT];
    int val;

    printf("UART RX ready.\n");

    while (1) {

        do {
            val = IORD_ALTERA_AVALON_PIO_DATA(NIOS_UARTRX_BASE) & 0x01;
        } while (val == 1);

        alt_timestamp_start();
        while (alt_timestamp() < (ticks_per_bit >> 1)) {}

        // Controllo sull'effettivo start
        val = IORD_ALTERA_AVALON_PIO_DATA(NIOS_UARTRX_BASE) & 0x01;
        if (val != 0) {
            continue;
        }

        alt_timestamp_start();
        int sample_times[NBIT + NSTOPBIT];
        for (int i = 0; i < NBIT + NSTOPBIT; i++)
            sample_times[i] = (i + 1) * ticks_per_bit;

        //-----------------------------------------
        // 4. Leggi i bit di DATA - loop unrolling

            while (alt_timestamp() < sample_times[0]) {}
            val = IORD_ALTERA_AVALON_PIO_DATA(NIOS_UARTRX_BASE) & 0x01;
            c[0] = val;

            while (alt_timestamp() < sample_times[1]) {}
            val = IORD_ALTERA_AVALON_PIO_DATA(NIOS_UARTRX_BASE) & 0x01;
            c[1] = val;

            while (alt_timestamp() < sample_times[2]) {}
            val = IORD_ALTERA_AVALON_PIO_DATA(NIOS_UARTRX_BASE) & 0x01;
            c[2] = val;

            while (alt_timestamp() < sample_times[3]) {}
            val = IORD_ALTERA_AVALON_PIO_DATA(NIOS_UARTRX_BASE) & 0x01;
            c[3] = val;

            while (alt_timestamp() < sample_times[4]) {}
            val = IORD_ALTERA_AVALON_PIO_DATA(NIOS_UARTRX_BASE) & 0x01;
            c[4] = val;

            while (alt_timestamp() < sample_times[5]) {}
            val = IORD_ALTERA_AVALON_PIO_DATA(NIOS_UARTRX_BASE) & 0x01;
            c[5] = val;

            while (alt_timestamp() < sample_times[6]) {}
            val = IORD_ALTERA_AVALON_PIO_DATA(NIOS_UARTRX_BASE) & 0x01;
            c[6] = val;

            while (alt_timestamp() < sample_times[7]) {}
            val = IORD_ALTERA_AVALON_PIO_DATA(NIOS_UARTRX_BASE) & 0x01;
            c[7] = val;

        while (alt_timestamp() < sample_times[NBIT]) {}
        val = IORD_ALTERA_AVALON_PIO_DATA(NIOS_UARTRX_BASE) & 0x01;

        if (val != 1) {
            printf("ERRORE: stop bit non valido!\n");
            continue;
        }

        int result = 0;
        for (int i = 0; i < NBIT; i++)
            // | è l'operatore OR bit a bit
            result |= (c[i] << i);

        printf("Ricevuto: %c (0x%02X) (%d) (%d%d%d%d%d%d%d%d)\n", result, result, result, c[7], c[6], c[5], c[4], c[3], c[2], c[1], c[0]);
    }

    return 0;
}
\end{minted}
\caption{Version\_2}
\label{lst:V2}
\end{listing}

Thanks to this approach the compiler avoids repeated branch instructions, making the code faster and more efficient than \ref{lst:V1}.
