\subsection{Project \#1}
To start the experience, the "Hello world" program is run.
\begin{minted}[linenos]{c}
#include <stdio.h>
#include "system.h"
#include "sys/alt_timestamp.h"
#include "altera_avalon_pio_regs.h"

int main()
{
  printf("Hello Sandro, Gianluca, Pietro and Cri!\n");

  return 0;
}
\end{minted}
The above program is running on the hardware described in VHDL files and flashed to the FPGA.\\
In order to make Software and Hardware able to work together is necessary the BSP. This work as an Hardware Abstraction Layer which is generated based on the instantiation of the different peripherals and components
with other informations, all available on the SOPC file, in this case the \textit{nios\_hps\_system.sopcinfo}.
