
\section{Introduction}

This lab deals with the creation, configuration, and execution of an embedded system predicated on the \textbf{Nios II soft-core processor} on the \textbf{Intel Cyclone V FPGA (DE1-SoC board)}. The main goal is to develop and implement an interface unit between the desktop and the FPGA system using the \textbf{UART (Universal Asynchronous Receiver-Transmitter)} protocol. Unlike higher level systems where the interface is abstracted away as an operating system driver, this project experimentally investigates the \textbf{"bit-banging"} and low-level hardware approach. 

The project uses the \textbf{General Purpose I/O (GPIO)} pins to sample and drive the wire line UART signals. This work attempts to explain the relation of software and hardware, specifically the execution of the software and the timing of the physical signals.

\subsection*{Educational Objectives}
The educational progression starts with a simple \textbf{"Hello World"} system and moves to more complex implementations, including a standalone \textbf{UART receiver} capable of functioning with a variety of baud rates. The design of the lab allows for the validation of the system and the study of various signals in real time using an \textbf{oscilloscope}. This enables the study of embedded real-time systems where timing and precise control of the processor are critical, allowing for the direct observation of \textbf{start/stop bits}, parity bits, and propagation delays.
