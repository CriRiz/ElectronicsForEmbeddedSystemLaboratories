\section{Introduction} 
The \textbf{FPGA (Field-Programmable Gate Array) design flow} is an extremely important concept to be understood given the transition from theoretical digital logic to real world implementation. Now, while an AND gate is the most simple and trivial digital building block, its implementation on a real commercial grade hardware such as the \textbf{Intel Cyclone V} is a multi-faceted problem as one has to take into account synthesis, physical placement and timing issues. The aim of this project is to gain a holistic perspective on the modern digital design cycle with the help of \textbf{VHDL/Verilog} and professional CAD design tools. In this, we seek to explore the true meaning of the relationship between \textbf{logical design} and \textbf{physical performance} as opposed to simplistic functional verification. 

\subsection*{Key Objectives} 
In the design, we undertake 3 major phases of design. They include: 

\begin{itemize} 
    \item \textbf{Functional Verification:} The logic is verified through the use of a testbench and simulation to confirm that the design is working. This is done through the development of an RTL description. 
    \item \textbf{Physical Implementation:} This involves the use of the \textit{Pin Planner} and \textit{Chip Planner} to perform logic distribution map and I/O block distribution map on the DE1-SoC board and to visualize the mapping of the design to the board. 
    \item \textbf{Timing Optimization:} This takes into account the \textbf{Propagation Delay} ($t_{pd}$) and involves trying different pin assignments and various timing constraints. 
\end{itemize}
As demonstrated in this project, the comparison of different hardware configurations like automatic and manual pin placement, and different distances between input-output pins shows how the physical routing and the rules of synthesis affect the performance and the robustness of the digital circuit.
