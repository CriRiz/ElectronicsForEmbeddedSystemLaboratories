\subsection{Project \#5}
This project deals again with the UART peripheral of the DE1-Soc board, this time polling technique was chosen. 
So as requested the baudrate was configured to 2400 and the interrupts were disabled through CONTROL register.
To monitor the device PuTTY can be used, without flow control and obiously configured to the same baudrate of UART(PuTTY config: 2400,8E1).

\subsubsection{Scenario \#1}
The code below can read the ASCII values present in RXDATA register and print the corresponding characters to stdout.
It can be noticed that it continuosly check the RRDY bit in the STATUS register in order to see if data are available.

\begin{minted}{c}
#define UART_0_BASE 0x8001060
#define UART_0_RX_REG (UART_0_BASE + 0)
#define UART_0_TX_REG (UART_0_BASE + 4)
#define UART_0_ST_REG (UART_0_BASE + 8)
#define UART_0_CNT_REG (UART_0_BASE + 12)
#define UART_0_DIV_REG (UART_0_BASE + 16)

#define RRDY_MASK       (1 << 7)   // Receive Ready
#define TRDY_MASK       (1 << 6)   // Transmit Ready

int main() {
  IOWR_ALTERA_AVALON_PIO_DATA(NIOS_HEADER_CONN_BASE, 0);
  volatile uint32_t *ptr_rx  = (volatile uint32_t *)UART_0_RX_REG;
  volatile uint32_t *ptr_tx  = (volatile uint32_t *)UART_0_TX_REG;
  volatile uint32_t *ptr_sts = (volatile uint32_t *)UART_0_ST_REG;
  volatile uint32_t *ptr_cnt = (volatile uint32_t *)UART_0_CNT_REG;
  volatile uint32_t *ptr_div = (volatile uint32_t *)UART_0_DIV_REG;

  //baudrate = 2400
  uint16_t divdata = 0x5160;
  *ptr_div = divdata;

  //disable interrupt
  *ptr_cnt = 0;

  printf("UART configured at 2400 baud, 8E1, polling.\n");

  while (1) {
      //print STATUS before reading
      printf("STATUS before RX: ");

      print_bin16((uint16_t)(*ptr_sts));
      //Polling: wait for RRDY to be set

      while (((*ptr_sts) & RRDY_MASK) == 0) {
          // busy wait
      }
      IOWR_ALTERA_AVALON_PIO_DATA(NIOS_HEADER_CONN_BASE, 1);
      //read RXDATA
      uint8_t rxchar = (uint8_t)(*ptr_rx & 0xFF);
      IOWR_ALTERA_AVALON_PIO_DATA(NIOS_HEADER_CONN_BASE, 0);

      //print STATUS after the reading
      printf("STATUS after RX:  ");
      print_bin16((uint16_t)(*ptr_sts));

      //print the received character
      printf("Received Character: '%c' (ASCII=%d)\n", rxchar, rxchar);
  }
  return 0;
}
\end{minted}
As shown above, the program prints the status register before and after the reading.

\begin{minted}{text}
  STATUS before RX: 0000000111111100
  STATUS after RX:  0000000101111100
  Received Character: 'g' (ASCII=103)
\end{minted}

\subsubsection{Scenario \#2}
Now the above exercise is repeated but this time assuming the RRDY bit of the status register is always set and so it is not necessary to check çit.

\begin{minted}{c}
#define RRDY_MASK       (1 << 7)   // Receive Ready
#define TRDY_MASK       (1 << 6)   // Transmit Ready

int main() {
  IOWR_ALTERA_AVALON_PIO_DATA(NIOS_HEADER_CONN_BASE, 0);
  volatile uint32_t *ptr_rx  = (volatile uint32_t *)UART_0_RX_REG;
  volatile uint32_t *ptr_sts = (volatile uint32_t *)UART_0_ST_REG;
  volatile uint32_t *ptr_cnt = (volatile uint32_t *)UART_0_CNT_REG;
  volatile uint32_t *ptr_div = (volatile uint32_t *)UART_0_DIV_REG;

  //baudrate 2400
  uint16_t divdata = 0x5160;
  *ptr_div = divdata;

  //Disable interrupt
  *ptr_cnt = 0;

  printf("UART configured at 2400 baud, 8E1, NO polling RRDY.\n");

  int freq = alt_timestamp_freq();
  const int DELAY_TICKS = freq/2;

  while (1) {

    IOWR_ALTERA_AVALON_PIO_DATA(NIOS_HEADER_CONN_BASE, 1);

    /*alt_timestamp_start();
    while (alt_timestamp() < DELAY_TICKS);
    */

    //read RXDATA without checking RRDY
    uint8_t rxchar = (uint8_t)(*ptr_rx & 0xFF);
    IOWR_ALTERA_AVALON_PIO_DATA(NIOS_HEADER_CONN_BASE, 0);

    //print received character
    printf("Received Character: '%c' (ASCII=%d)\n", rxchar, rxchar);
  }

  return 0;
}
\end{minted}

\subsubsection{Scenario \#3}