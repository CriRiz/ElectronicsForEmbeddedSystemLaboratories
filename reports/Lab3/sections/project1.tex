% !TEX root = ../main.tex

\subsection{Project \#1}

\begin{minted}{c}
int main() {

  uint16_t rxdata = *((volatile uint32_t *)UART_0_RX_REG) & 0xFFFF;
  uint16_t txdata = *((volatile uint32_t *)UART_0_TX_REG) & 0xFFFF;
  uint16_t stdata = *((volatile uint32_t *)UART_0_ST_REG) & 0xFFFF;
  uint16_t cntdata = *((volatile uint32_t *)UART_0_CNT_REG) & 0xFFFF;
  uint16_t divdata = *((volatile uint32_t *)UART_0_DIV_REG) & 0xFFFF;

  printf("RX DATA: ");
  print_bin16(rxdata);

  printf("TX DATA: ");
  print_bin16(txdata);

  printf("STATUS DATA: ");
  print_bin16(stdata);

  printf("CONTROLLER DATA: ");
  print_bin16(cntdata);

  printf("DIV DATA: ");
  print_bin16(divdata);

  return 0;
}
\end{minted}
The printed registers are:

\begin{table}[h]
	\centering
	\begin{tabular}{l l}
		\hline
		Register        & Value            \\
		\hline
		RX DATA         & 0000000000000000 \\
		TX DATA         & 0000000000000000 \\
		STATUS DATA     & 0000000001100000 \\
		CONTROLLER DATA & 0000000010000000 \\
		DIV DATA        & 0000000110110001 \\
		\hline
	\end{tabular}
	\caption{UART register values observed during the experiment}
\end{table}
These values can be compared with the configuration shown in Figure~\ref{fig:uart_config},
