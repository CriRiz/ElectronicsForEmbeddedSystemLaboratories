\subsection{Project \#4}
\subsubsection{Scenario \#1}
This project implements multiple chars transmissions. Previously, messages were sent without waiting the TRDY flag in the control register, which made the procedure prone to communication errors.

\begin{minted}[linenos]{c}
int main() {
  char msg[50] = "My name is Gianluca";

  volatile uint32_t *ptr_tx = (volatile uint32_t *)UART_0_TX_REG;
  volatile uint32_t *ptr_sts = (volatile uint32_t *)UART_0_ST_REG;

  // Configuration

  // The UART core's parity, data bits and stop bits are configurable.
  // These settings are fixed at system generation time; they cannot
  // be altered via the register file !!

  *ptr_div |= (uint16_t)0x5160; // This configures BR to 2400

  printf("Status before transmission: ");
  print_bin16((uint16_t)*ptr_sts);

  for (int i = 0; i < strlen(msg); i++) {

    // Wait trdy
    while (((*ptr_sts) & (1 << 6)) == 0);

    // Send msg
    *ptr_tx = (uint32_t)(msg[i]);
  }

  printf("Status after transmission: ");
  print_bin16((uint16_t)*ptr_sts);

  return 0;
}
\end{minted}
By setting the time scale to 10 ms, as shown below, the entire UART transmission is visible (\ref{fig:proj4})
