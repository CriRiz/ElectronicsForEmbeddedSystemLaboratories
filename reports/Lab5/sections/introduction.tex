\section{Introduction}
This last laboratory session deals with PWM signals and their possible applications.
The Role of PWM signals is to drive a device, for instance a LED, and regulate its behaviour by changing the Duty cycle of the input signal coming from the PWM generator.
The PWM system used in the laboratory consists of:
\begin{itemize}
  \item 12bit prescaler
  \item 8bit counter 
  \item 8bit register
  \item comparator
\end{itemize}
A 12bit prescaler divides a clock, based on the DIVISOR value, in our case divides the 50MHz clock of the DE1-SoC and change the frequency of the PWM.
The counter can be initialized to a maximum value called MAXCNT, its content is compared with the value contained in the register.
Then the PWM output is given following the result of the comparator, when the value on the counter is less than the value on the register it outputs a '1' else '0'.
Usefull formula below:

$$f_{\text{counter}} = \frac{f_{\text{clock}}}{\text{DIVISOR} + 1}$$
$$f_{\text{PWM}} = \frac{f_{\text{counter}}}{\text{MAXCNT} + 1} = \frac{f_{\text{clock}}}{(\text{DIVISOR} + 1)(\text{MAXCNT} + 1)}$$
$$t_{\text{PWM}} = \frac{1}{f_{\text{PWM}}}$$\\
Since the PWM system is described in vhdl and is running on the DE1-SoC, the commands of it are mapped on the switches and buttons present on the board.
The Exercises handled in this laboratory section are:
\begin{enumerate}
  \item PWM signal check
  \item Drive an LED with a BJT transistor
  \item Transistor delays
\end{enumerate}